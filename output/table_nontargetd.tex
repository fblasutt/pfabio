\begin{table}[htbp]\begin{threeparttable}\caption{Non-targeted moments}\label{table:nontargeted_moments}\centering\footnotesize\begin{tabular}{lcc} \toprule  Effect of the reform on &   Data & Model  \\\midrule    Behavioral pension points   & 0.10 &0.057\\ Work full time    & 0.05 &-0.004\\ Marginal employment    & -0.12 &-0.133\\ Non-marginal employment earnings (\euro)    & 2809 &1583\\Employed    & 0.10 &0.084\\\toprule    Other moments &   Data & Model  \\\midrule    Marginal propensity to earn (MPE)      & -0.51\text{ to }-0.12 &-0.33\\  \bottomrule\end{tabular}\begin{tablenotes}[flushleft]\small\item \textsc{Notes:} The numbers related to the effect of the reform in the data are the DiD coefficients reported in Tables \ref{pension_table:main_outcomes} and \ref{pension_table:pension_contrib}. The model counterparts are obtained by running the same DiD models as in the empirical section, but using simulated data.\\\end{tablenotes}\end{threeparttable}\end{table}