\begin{table}[htbp]\begin{threeparttable}\caption{Lifecycle model: counterfactual experiments}\label{table:experiments}\centering\footnotesize\begin{tabular}{lccccc} \toprule & Pension & Women's labor & Women's labor & Average age &  Welfare gains  \\&gender gap &hours &  participation  (\%) & at retirement  & wrt baseline (\%)  \\\midrule    Baseline                                   &0.508&16.28&68.73&66.72& 0.0\\ Caregiver credits                          &0.482&16.81&70.08&66.52&0.156\\ Caregiver credits, no threshold            &0.452&17.07&70.43&66.40&0.367\\ Lower income taxes                         &0.495&17.24&70.86&66.57&0.487\\ \bottomrule\end{tabular}\begin{tablenotes}[flushleft]\small\item \textsc{Notes:} The experiments in the last three rows imply the same government deficit. Welfare gains = increase in consumption at baseline to be indifferent with the experiment under analysis. Reforms are in place while the child is 10 y.o. or younger.\\\end{tablenotes}\end{threeparttable}\end{table}