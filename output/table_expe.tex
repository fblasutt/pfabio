\begin{table}[htbp]\begin{threeparttable}\caption{Lifecycle model: counterfactual experiments}\label{table:experiments}\centering\footnotesize\begin{tabular}{lcccc} \toprule & Pension &  Women's labor & Average age &  Welfare gains  \\&gender gap &hours &  at retirement  & wrt baseline (\%)  \\\midrule    Baseline                                   &0.400&16.55&66.51& 0.0\\ Caregiver credits                          &0.370&17.17&66.39&0.216\\ Caregiver credits, no threshold            &0.334&17.53&66.29&0.457\\ Lower income taxes                         &0.385&17.54&66.42&0.487\\ \bottomrule\end{tabular}\begin{tablenotes}[flushleft]\small\item \textsc{Notes:} The experiments in the last three rows imply the same government deficit. Welfare gains = increase in consumption at baseline to be indifferent with the experiment under analysis. Reforms are in place while the child is 3-10 y.o.\\\end{tablenotes}\end{threeparttable}\end{table}