\begin{table}[htbp]\begin{threeparttable}\caption{Lifecycle model: counterfactual experiments}\label{table:experiments}\centering\footnotesize\begin{tabular}{lcccc} \toprule & Pension & Women's labor & Average age &  Welfare gains  \\&gender gap &hours &  at retirement  & wrt baseline (\%)  \\\midrule    Baseline                                   &0.416&15.20&65.21& 0.0\\ Caregiver credits                          &0.394&15.51&65.07&0.244\\ Caregiver credits, no threshold            &0.395&15.52&65.09&0.244\\ Lower income taxes                         &0.408&15.75&65.18&0.582\\ \bottomrule\end{tabular}\begin{tablenotes}[flushleft]\small\item \textsc{Notes:} The experiments in the last three rows imply the same government deficit. Welfare gains = increase in consumption at baseline to be indifferent with the experiment under analysis. Reforms are in place while the child is 10 y.o. or younger\\\end{tablenotes}\end{threeparttable}\end{table}