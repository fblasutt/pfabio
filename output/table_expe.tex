\begin{table}[htbp]\begin{threeparttable}\caption{Lifecycle model: counterfactual experiments}\label{table:experiments}\centering\footnotesize\begin{tabular}{lccccc} \toprule & Pension & Women's labor & Women's labor & Average age &  Welfare gains  \\&gender gap &hours &  participation  (\%) & at retirement  & wrt baseline (\%)  \\\midrule    Baseline                                   &0.407&16.80&67.52&36.07& 0.0\\ Caregiver credits                          &0.375&17.45&68.28&35.96&0.286\\ Caregiver credits, no threshold            &0.367&17.46&68.01&35.96&0.286\\ Lower income taxes                         &0.386&18.05&69.03&35.99&0.827\\ \bottomrule\end{tabular}\begin{tablenotes}[flushleft]\small\item \textsc{Notes:} The experiments in the last three rows imply the same government deficit. Welfare gains = increase in consumption at baseline to be indifferent with the experiment under analysis. Reforms are in place while the child is 10 y.o. or younger\\\end{tablenotes}\end{threeparttable}\end{table}